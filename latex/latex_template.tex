% latex_template.tex

% This template creates an article with 12pt font
\documentclass[12pt]{article}
%\usepackage[margin=1in]{geometry}
%\newgeometry{left=.25in,right=.25in,top=.25in,bottom=.25in}
\usepackage{graphicx}
\usepackage{amsmath}
\usepackage{amssymb}
\usepackage{outlines}
\usepackage{enumitem}
\usepackage{epstopdf}
\usepackage{hyperref}
\usepackage[section]{placeins}
\usepackage{multirow}
\usepackage{csquotes}
\usepackage{placeins}
\usepackage{subfloat}
\usepackage{chngcntr}
\usepackage{natbib}
\usepackage{arydshln}
\usepackage{bbm} % to get \mathbbm{1} command for the indicator function
\newcommand\cites[1]{\citeauthor{#1}'s\ (\citeyear{#1})}
%\usepackage{lineno}
%\linenumbers
%\usepackage[font=small,skip=-8pt]{caption}
\usepackage[font=small]{caption}
\usepackage[top=1in,bottom=1in]{geometry}
\newcommand{\scalesize}{.4} 
\newcommand{\scalesz}{.5} 

%\input{mbc_newcommands.tex}

%%%%%%%%%%
%%%%%%%%%%
%%%%%%%%%%
\begin{document}

%\vspace{-1.5in}
% The \vspace{-.5in} command moves the title .5 inches higher
% than where it would normally be.
\title{\vspace{-.5in}\LaTeX\ Template}
\date{\today}
\author{Eric Hardy}
\maketitle
\begin{abstract}
	This template contains code examples commonly used in academic papers.  
\end{abstract}

\section{Motivation}\label{sec:motivation}
Microsoft Word is sometimes referred to as a ``WYSIWYG'' editor: ``What You See Is What You Get''. In contrast, \LaTeX is a markup language.  A LaTeX document begins as a plain text file, usually with a \texttt{.tex} file extension, in which users type commands to specify the layout and style of the document.  This text file is later compiled into a \texttt{.pdf} file which contains the formatted document.  

For example, write in \emph{italics} using \texttt{\char`\\ emph\{italics\}}, in \textbf{bold} using \texttt{\char`\\ textbf\{bold\}}, and in \texttt{monospaced} font using \texttt{\char`\\ texttt\{monospaced\}}.  

A markup language requires the user to learn some new commands to produce their desired formatting, but simultaneously gives the user a greater degree of control over the final document.  


\section{Math and Aligned Equations}\label{sec:math}
LaTeX excels at typesetting math.  Equations can be placed inline $y_t = A_tk_t^\alpha h_t^{1-\alpha}$, or as a standalone equation
A single equation 
\begin{equation}\label{eq:singleeqn}
	1 = a
\end{equation}

A matrix
\begin{equation}
	A = 
\begin{bmatrix}
	a & b \\
	c & d
\end{bmatrix}
\end{equation}
Cases, and plain text in math mode.  Spacing in math mode can be added with \texttt{\char`\\ }.  
\begin{equation}\label{eq:cases}
	B = 
	\begin{cases}
		b_1 \ \ \text{if} \ \ \alpha > 0 \\
		b_2 \ \ \text{if} \ \ \alpha \leq 0 \\
	\end{cases}
\end{equation}
% \underbrace{}_{}
% \left[\right]
% //\begin{bmatrix}
% bold in math mode \usepackage{amsmath}; \usepackage{bm} $\boldsymbol{K^b}$


% \subsection{}\label{ssec:}
% \subsubsection{}\label{sssec:}
% \footnote{}

to see a list of greek letters and math symbols: \url{http://web.ift.uib.no/Teori/KURS/WRK/TeX/symALL.html}
A single un-numbered equation
\begin{equation*}
	1 = b 
\end{equation*}
A single equation on multiple lines
\begin{equation}\begin{split}\label{eq:spliteqn}
	a &= b \\
	  &+ c
\end{split}\end{equation}

Multiple aligned equations
\begin{align}
	1 &= a + b \\
	2 &= a - b
\end{align}
the \texttt{\char`\\end\{align\}} command must come directly after the last equation in the \texttt{align} environment with no empty lines between them.  



	\section{Outlines}
Create an outline
\begin{outline}
\itemsep0em
	\1 Stuff
		\2 Note about Stuff
	\1 Other Stuff
\end{outline}
or an enumerated outline
\begin{outline}[enumerate]
	\1 Stuff
		\2 Note about Stuff
	\1 other Stuff
\end{outline}

\section{Tables}


\iffalse
\begin{subtables}
\input{raw/t1a_040718.tex}
\input{raw/t1b_040718.tex}
\end{subtables}

\begin{subtables}
\input{raw/t2a_040718.tex}
\input{raw/t2b_040718.tex}
\end{subtables}
\fi

\section{Figures}
%\clearpage

\bibliographystyle{chicago}
\bibliography{mcp_bibliography.bib}


\clearpage
\appendix 
\counterwithin{table}{section}
\section{Appendix Section}\label{sec:appendixsection}

\section{Additional Appendix Section}\label{sec:additionalappendixsection}


\end{document}
%%%%%%%%%%%%%%%%%%%%%%%%%%%%%%
%%%%%%%%%%%%%%%%%%%%%%%%%%%%%%
%%%%%%%%%%%%%%%%%%%%%%%%%%%%%%




%\nopagebreak
%\newgeometry{left=.3in,right=.3in,top=.3in,bottom=1in}
%\restoregeometry
%\nopagebreak


